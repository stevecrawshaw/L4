% Options for packages loaded elsewhere
\PassOptionsToPackage{unicode}{hyperref}
\PassOptionsToPackage{hyphens}{url}
\PassOptionsToPackage{dvipsnames,svgnames,x11names}{xcolor}
%
\documentclass[
  letterpaper,
  DIV=11,
  numbers=noendperiod]{scrartcl}
\usepackage{amsmath,amssymb}
\usepackage{lmodern}
\usepackage{iftex}
\ifPDFTeX
  \usepackage[T1]{fontenc}
  \usepackage[utf8]{inputenc}
  \usepackage{textcomp} % provide euro and other symbols
\else % if luatex or xetex
  \usepackage{unicode-math}
  \defaultfontfeatures{Scale=MatchLowercase}
  \defaultfontfeatures[\rmfamily]{Ligatures=TeX,Scale=1}
\fi
% Use upquote if available, for straight quotes in verbatim environments
\IfFileExists{upquote.sty}{\usepackage{upquote}}{}
\IfFileExists{microtype.sty}{% use microtype if available
  \usepackage[]{microtype}
  \UseMicrotypeSet[protrusion]{basicmath} % disable protrusion for tt fonts
}{}
\makeatletter
\@ifundefined{KOMAClassName}{% if non-KOMA class
  \IfFileExists{parskip.sty}{%
    \usepackage{parskip}
  }{% else
    \setlength{\parindent}{0pt}
    \setlength{\parskip}{6pt plus 2pt minus 1pt}}
}{% if KOMA class
  \KOMAoptions{parskip=half}}
\makeatother
\usepackage{xcolor}
\usepackage{longtable,booktabs,array}
\usepackage{calc} % for calculating minipage widths
% Correct order of tables after \paragraph or \subparagraph
\usepackage{etoolbox}
\makeatletter
\patchcmd\longtable{\par}{\if@noskipsec\mbox{}\fi\par}{}{}
\makeatother
% Allow footnotes in longtable head/foot
\IfFileExists{footnotehyper.sty}{\usepackage{footnotehyper}}{\usepackage{footnote}}
\makesavenoteenv{longtable}
\usepackage{graphicx}
\makeatletter
\def\maxwidth{\ifdim\Gin@nat@width>\linewidth\linewidth\else\Gin@nat@width\fi}
\def\maxheight{\ifdim\Gin@nat@height>\textheight\textheight\else\Gin@nat@height\fi}
\makeatother
% Scale images if necessary, so that they will not overflow the page
% margins by default, and it is still possible to overwrite the defaults
% using explicit options in \includegraphics[width, height, ...]{}
\setkeys{Gin}{width=\maxwidth,height=\maxheight,keepaspectratio}
% Set default figure placement to htbp
\makeatletter
\def\fps@figure{htbp}
\makeatother
\setlength{\emergencystretch}{3em} % prevent overfull lines
\providecommand{\tightlist}{%
  \setlength{\itemsep}{0pt}\setlength{\parskip}{0pt}}
\setcounter{secnumdepth}{-\maxdimen} % remove section numbering
% Make \paragraph and \subparagraph free-standing
\ifx\paragraph\undefined\else
  \let\oldparagraph\paragraph
  \renewcommand{\paragraph}[1]{\oldparagraph{#1}\mbox{}}
\fi
\ifx\subparagraph\undefined\else
  \let\oldsubparagraph\subparagraph
  \renewcommand{\subparagraph}[1]{\oldsubparagraph{#1}\mbox{}}
\fi
\KOMAoption{captions}{tableheading}
\makeatletter
\makeatother
\makeatletter
\@ifpackageloaded{caption}{}{\usepackage{caption}}
\AtBeginDocument{%
\renewcommand*\contentsname{Table of contents}
\renewcommand*\listfigurename{List of Figures}
\renewcommand*\listtablename{List of Tables}
\renewcommand*\figurename{Figure}
\renewcommand*\tablename{Table}
}
\@ifpackageloaded{float}{}{\usepackage{float}}
\floatstyle{ruled}
\@ifundefined{c@chapter}{\newfloat{codelisting}{h}{lop}}{\newfloat{codelisting}{h}{lop}[chapter]}
\floatname{codelisting}{Listing}
\newcommand*\listoflistings{\listof{codelisting}{List of Listings}}
\makeatother
\makeatletter
\@ifpackageloaded{caption}{}{\usepackage{caption}}
\@ifpackageloaded{subcaption}{}{\usepackage{subcaption}}
\makeatother
\makeatletter
\@ifpackageloaded{tcolorbox}{}{\usepackage[many]{tcolorbox}}
\makeatother
\makeatletter
\@ifundefined{shadecolor}{\definecolor{shadecolor}{rgb}{.97, .97, .97}}
\makeatother
\makeatletter
\makeatother
\ifLuaTeX
  \usepackage{selnolig}  % disable illegal ligatures
\fi
\IfFileExists{bookmark.sty}{\usepackage{bookmark}}{\usepackage{hyperref}}
\IfFileExists{xurl.sty}{\usepackage{xurl}}{} % add URL line breaks if available
\urlstyle{same} % disable monospaced font for URLs
\hypersetup{
  pdftitle={Industry Insights: John Walton},
  pdfauthor={Steve Crawshaw},
  colorlinks=true,
  linkcolor={blue},
  filecolor={Maroon},
  citecolor={Blue},
  urlcolor={Blue},
  pdfcreator={LaTeX via pandoc}}

\title{Industry Insights: John Walton}
\author{Steve Crawshaw}
\date{}

\begin{document}
\maketitle

\ifdefined\Shaded\renewenvironment{Shaded}{\begin{tcolorbox}[enhanced, borderline west={3pt}{0pt}{shadecolor}, breakable, sharp corners, frame hidden, boxrule=0pt, interior hidden]}{\end{tcolorbox}}\fi

\hypertarget{summary}{%
\section{Summary}\label{summary}}

John is editor of data journalism at the BBC. This was an interesting
talk on the application of data science to the world of journalism. John
highlighted several projects at the BBC where data added insight to
stories, or where the analysis of the data created the story. He gave
advice on starting a career in data journalism and how to apply data
skills in story telling.

\hypertarget{data-journalism}{%
\section{Data Journalism}\label{data-journalism}}

More data are available nowadays and data skills are becoming more
necessary than ever. Data journalists need to see the stories in the
data so need data skills to do that. Data journalists need to train each
other on the job - a virtuous circle. This can be rewarding and is also
necessary to QA the data and for effective processing. Collaborative
approach, working with many disciplines The journalistic skills are
essential - inquisitive and critical.

\hypertarget{projects}{%
\subsection{Projects}\label{projects}}

\hypertarget{covid---19}{%
\subsubsection{Covid - 19}\label{covid---19}}

\begin{itemize}
\tightlist
\item
  BBC page covering COVID stats for 15 months
\item
  A data dashboard basically - but no data links..
\item
  Taking official figures and updating daily
\item
  Trends - deaths, cases, graphics
\item
  Postcode search is a key attribute
\end{itemize}

\hypertarget{climate-change}{%
\subsubsection{Climate change}\label{climate-change}}

\begin{itemize}
\tightlist
\item
  \href{https://www.bbc.co.uk/news/resources/idt-d6338d9f-8789-4bc2-b6d7-3691c0e7d138}{How
  CC may affect you in your area}
\item
  Collaboration with Met office and Panorama
\end{itemize}

\hypertarget{nhs-waiting-lists}{%
\subsubsection{NHS Waiting Lists}\label{nhs-waiting-lists}}

\begin{itemize}
\tightlist
\item
  Use people's stories and combine with NHS open data
\item
  Explain the impact of COVID-19 on this metric
\end{itemize}

\hypertarget{uk-passport-checker-shows-skin-colour-bias}{%
\subsubsection{UK passport checker shows skin colour
bias}\label{uk-passport-checker-shows-skin-colour-bias}}

\begin{itemize}
\tightlist
\item
  Higher rate of photo rejection on automated checking site
\item
  tested \textgreater{} 1000 photos automatically to identify bias
\end{itemize}

\hypertarget{fast-fashion-online-advertising-is-more-sexual}{%
\subsubsection{Fast fashion online advertising is more
sexual}\label{fast-fashion-online-advertising-is-more-sexual}}

\begin{itemize}
\tightlist
\item
  Machine learning using a cohort of photos downloaded from online
  advertising
\end{itemize}

\hypertarget{questions}{%
\subsection{Questions}\label{questions}}

\begin{itemize}
\tightlist
\item
  How will data change journalism?

  \begin{itemize}
  \tightlist
  \item
    Not all journalists will need to code. Data analysis is a natural
    fit for journalists but not all journalists will need data skills
  \end{itemize}
\item
  Will there be more jobs for Data journalists

  \begin{itemize}
  \tightlist
  \item
    Quite likely, esp with increased data literacy engendered by
    COVID-19
  \end{itemize}
\item
  How do you become a data journalist?

  \begin{itemize}
  \tightlist
  \item
    Need to show that you can use data, or have atrack record with data
    stories.
  \end{itemize}
\item
  Advice to Potential data journalists?

  \begin{itemize}
  \item
    Acquire data skills while studying rather than working.
  \item
    Hurdles will depend on where you are in your career.
  \item
    Follow your own ideas about stories from data.
  \end{itemize}
\item
  Data first or story first?

  \begin{itemize}
  \item
    Depends. Can plan for release of key datasets
  \item
    Time is a key constraint and will limit depth of data analysis
  \end{itemize}
\item
  Is there potential for broadcast news timescale data stories?

  \begin{itemize}
  \tightlist
  \item
    Yes if planning is done, but difficult to be reactive.
  \end{itemize}
\item
  Is analysis or data viz the most time consuming?

  \begin{itemize}
  \item
    They both go together, data viz informs analysis
  \item
    Use R for analysis and dataviz rather than separate software
  \end{itemize}
\item
  How do you ensure data stories aren't biased?

  \begin{itemize}
  \item
    Need to try and cover significant and interesting aspects
  \item
    Speak to people who provide the data
  \end{itemize}
\item
  How to identify gaps in data?

  \begin{itemize}
  \tightlist
  \item
    Sometimes have to accept data is imperfect and acknowledge it
  \end{itemize}
\item
  Why just static data viz?

  \begin{itemize}
  \item
    Sometimes interactivity can be overkill
  \item
    May not be appropriate for some formats (phones) or users
  \end{itemize}
\item
  How to target your audience and explain concepts like risk?

  \begin{itemize}
  \item
    It's a general audience, so everyone.
  \item
    Risk - Winton Centre, advice from Spiegelhalter
  \end{itemize}
\item
  Do you share best practice?

  \begin{itemize}
  \tightlist
  \item
    Yes, also with BBC head of Statistics
  \end{itemize}
\end{itemize}

\end{document}
